We developed two python scripts that plot useful graphs to quickly visualize the system performance, and 
that compute error metrics to objectively compare different test instances. 
\subsection{Metrics}
Let $\vec{q(t)}=\begin{bmatrix} x(t) \\ y(t) \end{bmatrix}$ be the position of the robot center of mass in global coordinates at 
time t, and $\vec{q_{ref}(t)}=\begin{bmatrix} x_{ref}(t) \\ y_{ref}(t) \end{bmatrix}$ be its target position at time t, we define
the error vector $\vec{e}$ at time t as:
\[
    \vec{e(t)} = \begin{bmatrix} x(t) - x_{ref}(t) \\ y(t) - y_{ref}(t) \end{bmatrix}
\] 
We neglect the heading $\theta$ in the error computation since, having used the feedback linearization law to control
the system, we have a loss of observability: we no longer can control the robot heading; however, we know that, starting
from any initial condition, the car will automatically align itself with the desired trajectory. \\
To quantify the controlled system performance we chose the following two indicators:
\begin{itemize}
    \item Root Mean Square Error (RMSE) which is computed as the square root of the mean square error:
    \[
        RMSE = \sqrt{\frac{1}{T} \sum_{t=0}^{T}\vec{e(t)}^\intercal\vec{e(t)}}
    \] 
    It is a measure of the mean error that we have at each time instant;
    \item Integral Square Error (ISE) which is computed as the integration over the entire trajectory time T of the square
    of the error:
    \[
        ISE = \int_{T}^{} \vec{e(t)}^\intercal\vec{e(t)}\,dt
    \] 
    It is a measure of the total error accumulated over the entire trajectory.
\end{itemize}

\subsection{Graphs}
\textit{analyze\_bag.py} script plots three graphs:
\begin{itemize}
    \item  a 2D representation of the robots center of mass trajectory (continuous blue line) together with the 
    desired trajectory (dotted orange line);
    \item actual and desired center of mass trajectory along x-axis over time;
    \item actual and desired center of mass trajectory along y-axis over time.
\end{itemize}
When using Pacejka tire-ground interaction model, we can use \textit{analyze\_speed\_bag.py} script that draws the three
aforementioned plots plus two more:
\begin{itemize}
    \item the longitudinal force $F_x$ acting on the wheel contact point with respect to the longitudinal slip;
    \item the lateral force $F_y$ acting on the wheel contact point with respect to the lateral slip.
\end{itemize}

\subsection{Scripts usage}
The typical use case of the python scripts is to, first, record a ROS bag during the simulation; it can be done using command:
\begin{lstlisting}[language=bash]
    $ rosbag record -O out.bag --duration=1m /vesc/odom
        /reference_trajectory
\end{lstlisting}
Or, if we are using Pacejka tire-ground interaction model, we need to record additional data with the following:
\begin{lstlisting}[language=bash,  label={pacejka}]
    $ rosbag record -O out.bag --duration=1m /vesc/odom 
        /reference_trajectory /long_pub/right_front 
        /lat_pub/right_front /fx_pub/right_front
        /fy_pub/right_front
\end{lstlisting}
Then, we can simply launch the desired python script passing the bag file as parameter:
\begin{lstlisting}[language=bash]
    $ python3 analyze_bag.py -O out.bag
\end{lstlisting}
or 
\begin{lstlisting}[language=bash]
    $ python3 analyze_speed_bag.py -O out.bag
\end{lstlisting}
Please note that in order to use \textit{analyze\_speed\_bag.py}, the bag file must have been recorded using the command
\hyperref[pacejka]{above}.