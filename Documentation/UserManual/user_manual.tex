\documentclass[12pt, letterpaper, oneside]{report}

\usepackage[a4paper, portrait, margin=1in]{geometry}
\usepackage[utf8]{inputenc}
\usepackage[english]{babel}
\usepackage{concmath} % Text Font
\usepackage[T1]{fontenc}
\usepackage{tabularx} % Manage Tables
\usepackage{fancyhdr} % For Headers and Footers
%\usepackage[htt]{hyphenat} % To solve \textit line break problem

\title{"Control of Mobile Robot" Project}
\author{Colli Stefano, Pagani Mattia, Panelli Erica}
\date{\today}

\pagestyle{fancy}
\fancyhf{}
\rhead{CMR Project}
\lhead{User Manual}
\rfoot{\thepage}

\begin{document}
	
\maketitle

\begin{abstract}
\end{abstract}

\tableofcontents

\newpage

%\section{Gazebo Body Parameters}
%
%\begin{itemize}
%	\item $\mu_0$: friction coefficient of direction 1
%	\item $\mu_1$: friction coefficient of direction 2
%	\item fdir: specify direction of $\mu_0$, direction of $\mu_1$ is perpendicular to this parameter
%	\item $k_p$: stiffness between bodies
%	\item $k_d$: dumping between bodies
%\end{itemize}

\chapter{Notes on Installation and Launch}

\section{Installation}

\subsection{Downloading material}

\subsection{Additional packages to be installed}

\subsection{Additional modifications}

\section{Launch}

\chapter{General Project Structure}

\section{Catkin Workspace Directories}

\subsection{Original MIT Racecar Packages}

\begin{center}
	\begin{tabularx}{\textwidth}{
			| >{\raggedright\arraybackslash}X
			| >{\arraybackslash}X |
		}
		\hline
		ackermann\_cmd\_mux & ... \\
		\hline
		ackermann\_msgs & Contains definitions of \textbf{AckermannDrive} and \textbf{AckermannDriveStamped} messages, used by the racecar to compute movements. \\
		\hline
		racecar & ... \\
		\hline
		racecar\_control & Contains launch files to load controllers used to manage the motors of the racecar. Also load nodes which dispatch messages to controllers. \\
		\hline
		racecar\_description & Contains a description of the racecar, in terms of models, meshes ecc... It will be used by Gazebo to represent it. \\
		\hline
		racecar\_gazebo & Mainly contains launch scripts used to load all necessary nodes, worlds and other components to open a Gazebo instance with a controllable car. \\
		\hline
	\end{tabularx}
\end{center}

\subsection{Added Packages}

\begin{center}
	\begin{tabularx}{\textwidth}{
			| >{\raggedright\arraybackslash}X
			| >{\raggedright\arraybackslash}X |
		}
		\hline
		car\_control & Contains nodes used to send commands to the racecar, receive odometry data and control the car autonomously. \\
		\hline
	\end{tabularx}
\end{center}

\chapter{Original System Introduction}
\chapter{(Our) System Description}
\chapter{Detailed Package Description}

\section{Package car\_control}

\subsection{Node car\_commands\_node}

This node has been thought to be an interface between the racecar and the controller. It should adapt commands received by the controller, which are in a specific format, and transform them in \textbf{AckermannDriveStamped} messages.

Parameters of the node, as initial velocity and initial steering angle, are written in a specific YAML configuration file.

\subsubsection{Actual Implementation}
This node simply send an \textbf{AckermannDriveStamped} message to the topic \textit{vesc\slash ackermann\_cmd\_mux\slash input\slash teleop}. That topic is used by the original MIT system to read commands to perform computation to make the racecar move.

The sending criterion is simple. It just generate messages which make the racecar move in a straight line at a specific velocity.

\end{document}