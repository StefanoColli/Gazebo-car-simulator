\subsection{Intro}

We have added this plugin to allow the model to simulate the interaction between the wheels and the underlying terrain using the Pacejka Magic Formula.
The interaction between the material of the wheels and the road surface gives rise to a deformation of the wheel contact surface and consequently to forces and torques acting on the vehicle: there are different models to define these quantities resulting from tire-road interactions. We have implemented the Magic Formula defined by Hans B. Pacejka.

\subsection{Pacejka Magic Formula}
The tire-road interaction gives rise to six resultants:
\begin{itemize}
	\item $F_x$ - Longitudinal Force
	\item $F_y$ - Lateral Force
	\item $F_z$ - Normal Force (Vertical Load)
	\item $M_x$ - Overturning Moment
	\item $M_y$ - Rolling Resistance Moment
	\item $M_z$ - Self-aligning torque
\end{itemize}
Within our plugin, we have only calculated the resulting $F_x$ and $F_y$ forces. The reason for this choice stems from the adjustable parameters in Gazebo: for each calculated velocity and steering angle, the $slip1$ and $slip2$ parameters are modified, corresponding to the amount of slip in the two reference directions $x$, $y$ of the vehicle. These values depend on the forces acting on that portion of the contact surface at a given moment in time.

\subsubsection{Longitudinal force $F_x$}
The longitudinal force according to the Pacejka Magic Formula depends on the longitudinal slip, which determines how far the wheel deviates from pure slipping behavior during a longitudinal displacement (i.e., a displacement that depends only on the $x$-axis).
\\
The formula used to compute the longitudinal slip is:
\[
\sigma_x = 
  \begin{cases} 
	\frac{V_x - r_e\Omega}{V_x} & \text{if $V_x > r_e\Omega$} \\
	\frac{r_e\Omega - V_x}{r_e\Omega} & \text{otherwise}
  \end{cases}
\] \\

After calculating the longitudinal slip, we used the Pacejka formula to compute $F_x$.
\\
\[F_x = D_x\sin(C_x\arctan(B_x\sigma_x - E_x(B_x\sigma_x - \arctan(B_x\sigma_x))))\]\\

\subsubsection{Lateral Force $F_y$}
The lateral force according to the Pacejka Magic Formula depends on the lateral slip, which determines the angle between the direction in which the wheel is moving and the direction it is pointing.
\\
The formula used to compute the lateral slip is:
\[
\alpha = - \arctan{\frac{V_{c_y}}{V_{c_x}}}
\] \\
After calculating the lateral slip, we used the Pacejka formula to compute $F_y$.
\\
\[F_y = D_y\sin(C_y\arctan(B_y\alpha - E_y(B_y\alpha - \arctan(B_y\alpha))))\]\\

\subsubsection{Parameters of the Pacejka Magic Formula}
Applying the Pacejka Magic Formula also requires determining the parameters used within the formula itself. 
To calculate these parameters, we need to compute the wheel's normal force $F_z$: we have done this by considering the total weight of the vehicle and evenly distributing the entire weight among the four wheels.
\\
\[M_{tot} = M_{chassis} + 4M_{wheel} + M_{steering-hinge} + M_{laser} = \]
 \[ = 4.0 + 4*0.34055 + 0.100 + 0.130 = 5.5922 \ [kg]\]
\[g = 9.8065 \ [m/s^2]\]
\[F_z = \frac{M_{tot}g}{4} = 13.71 \ [N]\]\\

Initially, we roughly calculated the parameters B, C, D, and E using the theoretical specifications from the documentation related to the Magic Formula. Subsequently, we carried out an experimental tuning process of the parameters to enhance the model's performance.
For the sake of calculation simplicity, we assumed that the vehicle had no additional load beyond its own weight; therefore, the formulas used to calculate the parameters are simplified.

\begin{itemize}
	\item \textbf{$C$ Shape factor} \\
		  This parameter depends on the cross-section of the wheel's geometry. 
		  In the racecar model, the wheels are defined as cylinders with a radius \[R = 0.05 \ [m]\] and length \[L = 0.045 \ [m]\].
		  We compute the shape factor with the formula: \[ C = \frac{L+2R}{2R}\]
	\item \textbf{$D$ Peak factor} \\
		  The peak value of the force generated due to tire-road interaction.
		  \[ D = F_z(b_1 + F_zb_2)\]
		  Where:
		  \begin{itemize}
		  	\item $F_z$ is the wheel's normal force $[kN]$
		  	\item $b_1$ is the load influence on the longitudinal friction coefficient$(*1000)$ \\
		  	In our case $b_1 = 0$ (no load assumption).
		  	\item $b_2$ is the longitudinal friction coefficient$(*1000)$\\
		  	In our case $b_2$ varies depending on the considered wheel. The longitudinal friction coefficient and the reference direction are defined for each wheel within the $racecar.gazebo$ file ($mu_1$, $mu_2$, and $fdir_1$). Therefore, depending on whether the wheels are front or rear, we will have two values $D_x$ and $D_y$ with different values.
		  \end{itemize}
	  \item \textbf{$BCD$ Cornering stiffness} \\
	  		It is the slope at the origin of the curve between force and slip. It is particularly important to allow for a linear approximation of the curve.
	  		\[BCD = (b_3{F_z}^2) + (b_4F_z){e}^{-b_5F_z}\]
	  		Where:
	  		\begin{itemize}
	  			\item $F_z$ is the wheel's normal force $[kN]$
	  			\item $b_3$ is the curvature factor of stiffness/load \\
	  				  In our case $b_3 = 0$ (no load assumption).
	  			\item $b_4$ is the change of stiffness with slip $[N/\%]$ \\
	  				  This parameter concerns the change of stiffness with slip. It is a physical property, how much an object resists deformation. The stiffer the object is, the more resistant to deformation it is.
	  			\item $b_5$ is the change of progressivity of stiffness/load \\
	  				  In our case $b_5 = 0$ (no load assumption).
	  		\end{itemize}
  			We computed the parameter $BCD$ using the reference values provided by the Pacejka Magic Formula documentation and performing parameter tuning operations.
  	\item \textbf{$B$ Stiffness factor} \\
  			The stiffness factor is a measure of the resistance of a material to deformation. The higher the stiffness factor, the more resistant the material is to deformation.
  			\[B = \frac{BCD}{CD}\]
  			We computed the parameter $B$ using the reference values provided by the Pacejka Magic Formula documentation and performing parameter tuning operations.
  	\item \textbf{$E$ Curvature factor} \\
  			The parameter $E$ controls the shape of the nonlinear force curve with respect to the slip. It influences the curvature of the force curve and helps to model the asymmetry of tire behavior for positive and negative slips.
  			\[E = (b_6{F_z}^2 + b_7·F_z + b_8) · (1 - b_{13}sign(slip+H))\]
  			Where:
  			\begin{itemize}
  				\item $F_z$ is the wheel's normal force $[kN]$
  				\item $b_6$ is the ${curvature change with load}^2$ \\ 
  					  In our case $b_6 = 0$ (no load assumption).
  				\item $b_7$ is the curvature change with load \\
  					  In our case $b_7 = 0$ (no load assumption).
  				\item $b_8$ is the curvature factor
  				\item $b_{13}$ is the curvature shift
  				\item $H$ is the horizontal shift
  				\item $slip$ is the longitudinal or lateral slip factor
  			\end{itemize}
  			We computed the parameter $E$ using the reference values provided by the Pacejka Magic Formula documentation and performing parameter tuning operations.
\end{itemize}
Below, we have included the table with the values of all the parameters used within the formulas for the calculation of Fx and Fy, respectively.
\begin{center}
	\begin{tabular}{||c c c||} 
		\hline
		 & Longitudinal Force $[F_x]$ & Lateral Force $[F_y]$ \\ [0.5ex] 
		\hline\hline
		B & 3 & 10 \\ 
		\hline
		C & 1.45 & 1.45 \\
		\hline
		D & 19.0569 & 1.371 \\
		\hline
		E & 0.52 & 0.97 \\
		\hline
	\end{tabular}
\end{center}

\subsection{Plugin description}

Our goal was to modify the physical parameters that govern the dynamics of the interaction between the vehicle and the terrain in real time during the simulation.
To achieve this, we had to directly modify the parameters in the simulator using the updated velocities and angles in the ROS model. It was necessary to develop a plugin that could receive information from the model's controller and communicate it in real-time to the simulator.
After some research, we have decided to use it as a base and modify the two plugins, $GazeboRosWheelSlip$ and $WheelSlipPlugin$ developed to modify slip compliance in the two fundamental directions of a vehicle with one or more wheels.
While we kept the $GazeboRosWheelSlip$ plugin practically unchanged, we substantially modified $WheelSlipPlugin$. We have included all the computations necessary for the application of Pacejka formulas. For this reason, we wanted to differentiate the two plugins by changing the name, from $WheelSlipPlugin$ to $wheel\_slip\_pacejka$.

\subsubsection{Plugin configuration}
The plugin must be called within the configuration file, in our case $racecar.gazebo$.\\
For each wheel, we have assigned a $wheel$ tag in which, through sub-tags, we have passed the values that the plugin needs to perform all computations. \\
In particular, each $wheel$ tag includes:
\begin{itemize}
	\item \textbf{slip\_compliance\_lateral} \\
		  Unitless slip compliance $(slip / friction)$ in the lateral direction. 
	\item \textbf{slip\_compliance\_longitudinal} \\
		  Unitless slip compliance $(slip / friction)$ in the longitudinal direction. 
	\item \textbf{wheel\_normal\_force} \\
		  Normal force value impressed on the wheel.
	\item \textbf{wheel\_radius} \\
		  Radius $[m]$ of the cross-section of the wheel's geometry.
\end{itemize}
All these values are applied to all wheels declared in the Element\_Ptr passed firstly to $GazeboRosWheelSlip$ and secondly to $wheel\_slip\_pacejka$ plugin.

\subsubsection{GazeboRosWheelSlip}
This plugin was necessary to interface our ROS model with the Gazebo plugin wheel\_slip\_pacejka, and permit the exchange of values to update and modify the slip compliance values.
This plugin calls directly the $Load$ function of wheel\_slip\_pacejka, so it does not add any computation to the slip compliance, it only handles the communication between ROS and Gazebo.

\subsubsection{wheel\_slip\_pacejka}
This is a plugin that updates ODE wheel slip parameters based on linear wheel spin velocity $radius * spin\_rate$. The ODE slip parameter is documented as force-dependent slip ($slip_1$, $slip_2$) in the ODE user guide, it has units of $[velocity / force] = [m / s / N]$, similar to the inverse of a viscous damping coefficient.
The $slip\_compliance$ parameters specified in this plugin are unitless, representing the lateral or longitudinal slip ratio to tangential force ratio $tangential / normal\_force$.
At each time step, these compliances are multiplied by the linear wheel spin velocity and divided by the wheel\_normal\_force parameter specified below to match the units of the ODE slip parameters.
We used these formulas as already present in the $WheelSlipPlugin$, but with the forces computed by using Pacejka Magic Formula and the slip computed with the formulas presented above in this section.

